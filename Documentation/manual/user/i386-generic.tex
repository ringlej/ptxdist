% ----------------------------------------------------------------------------
\chapter{The i386-generic-glibc Target}		\label{chap:i386-generic}
% ----------------------------------------------------------------------------

Now, we have enough theoretical background to try out a first example.
The PTXdist developers have prepared a generic configuration which can
be run on every Intel 386 compatible PC that might happen to go to seed
in a dark corner of your lab. Consisting of "normal", non-embedded,
hardware, it can be used as an ideal starting point for your first
PTXdist experiments. Contrary to a real target, a normal PC has the
advantage of having standard interfaces, such as a keyboard or a video
card with a monitor, which makes it easier to find out why something
goes wrong in case of a problem. 

Select the generic target by entering at your bash command line

\begin{code}
robert@himalia:~/ptxdist> make i386-generic-glibc_config
copying i386-generic-glibc configuration
robert@himalia:~/ptxdist>
\end{code}

\begin{important}
If you have played around with the PTXdist tree you are working in
before, don't forget to run \texttt{make distclean} before this command.
Whenever you have the impression that something is strange with
your tree, distclean it and start from a well-known starting
point. 
\end{important}

PTXdist needs a directory where it can install it's host tools. By
default, this directory is set to \texttt{/tmp/ptxdist-local-generic},
which should not collide with any other program. Please check if this
directory is empty or non-existent on your host, to avoid conficts with
old stuff. When there is old stuff laying around in this directory,
either delete it (\texttt{rm -fr /tmp/ptxdist-local-generic}) after
carefully checking if you \emph{really} delete it, or chose another
directory for your host tools (see chapter~\ref{} to find out how to do
this). We refer to the directory containing the host tools as the
\texttt{PTXCONF\_PREFIX} directory from now on, as this is the name the
PTXdist build system uses internally. 

We assume everything is configured correctly now and run \texttt{make
oldconfig} to evaluate and process the new configuration now. You should
see some compiler output messages now, a bunch of questions are printed
out, including their respective answers, and the system should come back
to the shell command line prompt without stopping.

\begin{important}
If PTXdist stops while running \texttt{make oldconfig} and asks stupid
questions, the config file you are using doen't fit the currently used
PTXdist version. This probably means that it was not updated correctly
by the maintainer when he made the last release; write a mail with a bug
report to the PTXdist mailing list. 
\end{important}

Now, you need two things: first of all, either a CD with all sources
being necessary for your target, or an internet connection which lets
your development host get all the source packages from the net. As we
already discussed, PTXdist decides during runtime which packages are
needed to compile your root filesystem: the decision is based on what
the currently used configuration has selected and on implicit
dependencies which are resolved automatically. 

In case you choose the internet connection variant you don't have to do
anything and can start building, as described below. When you want to
avoid the source packet downloads, copy all the source archives from
your CD to the \texttt{src/} directory of the PTXdist tree. While
running PTXdist the system tries to extract packages when they are about
to being processed; when they are not available they are transferred via
HTTP or FTP connection, so when you copy everything to the \texttt{src/}
directory no archives are transferred. If you forgot one or the other
packet it is no problem, it just is pulled from the net in that case.
Take this into account when you have a low-bandwith or pay-per-volume
line for your internet traffic. 

We now start the main compiler run:

\begin{code}
robert@himalia:~/ptxdist> make world
\end{code}

This triggers the whole mechanism and if nothing goes wrong you get a
ready-to-use root filesystem for the embedded system after a while. If
you want to have more information about what happens during the build,
if you want to have a logfile for the case when something goes wrong and
if you want to find out how long the build-run took to complete use the
following line: 

\begin{code}
robert@himalia:~/ptxdist> time make world 2>&1 | tee logfile
\end{code}

The \texttt{time} command prints out the time the whole thing took,
until control comes back to the command line. With \texttt{2>\&1} you
redirect all the stderr-output to stdout, which is afterwards fed to the
\texttt{tee} command, so all the output messages are not only printed to
the terminal but also logged in \texttt{logfile}. In case of an error
\texttt{logfile} may contain useful information about what went wrong. 

If everything has completed \texttt{root/} does contain the target's
root filesystem. As the \emph{i386-generic-glibc} target was developed
for the normal Intel architecture it can now be directly tested: 

\begin{code}
root@himalia:/home/robert/ptxdist> chroot root/ /bin/sh
\end{code}

Note that the \texttt{chroot} command can only be executed when being
the root user. It locally changes the root path to the \texttt{root/}
directory for the target and executes, already from this new point of
view, the given binary (\texttt{/bin/sh} in this case). If you see a
prompt after that everything is ok: you just changed root into your new
self made "embedded system emulator".   

